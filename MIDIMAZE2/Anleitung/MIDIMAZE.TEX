\documentclass[12pt,twoside]{article}
\usepackage{graphicx}

\newcommand{\mmI}{{\sc Midi-Maze}}

\newcommand{\mm}{\mmI~{\sc II}}

\newcommand{\grad}{$^\circ$}

\newcommand{\bild}[2]{\begin{figure}[hbtp]%
\center\mbox{%
\includegraphics{#1}%
}\caption{#2}\label{ab@_#1}\end{figure}}

%deflist-Environment nach Dr. R. Wonneberger
\newcommand{\deflabel}[1]{\bf #1\hfill}%
\newenvironment{deflist}[1]%
{\begin{list}{}% Irgendeiner der Remarks ist NôTIG!
{\settowidth{\labelwidth}{\bf #1}%
\setlength{\leftmargin}{\labelwidth}%
\addtolength{\leftmargin}{\labelsep}%
\renewcommand{\makelabel}{\deflabel}}}%
{\end{list}}%


\pagestyle{myheadings}

\markboth{$\Sigma$-soft\hfil\mm}{\mm\hfil $\Sigma$-soft}


\begin{document}

\section{Willkommen zu \mm!}


\mm\ ist ein Spiel f\"ur bis zu 16~Spieler, die Ihre Computer miteinander \"uber 
MIDI vernetzen. Man braucht {\em unbedingt\/} mehr als einen Rechner! \mm\ 
l\"a\ss{}t sich {\em nur\/} mit mehr als einem Spieler spielen. Jeder Mitspieler 
steuert einen kugelf\"ormigen Smiley durch ein Maze\footnote{engl. Maze hei\ss{}t 
Labyrinth und gab diesem Spiel seinen Namen.}; wer einen Freund sieht, hilft 
ihm, wer einen Feind sieht, schie\ss{}t ihn ab (oder umgekehrt). Ein anfangs 
frischer Spieler kann mit dem 3.~Treffer erlegt werden. Soweit ist das Spiel 
mit \mmI\ identisch. Bei \mm\ gibt es noch Extras, die man sich f\"ur Punkte 
kaufen kann. Einzelheiten stehen weiter unten. Der Grundgedanke des Spiels 
ist also recht einfach; wir haben jedoch gemerkt, da\ss{} gerade dadurch und 
durch das zusammenspielen mit anderen der Spielspa\ss{} aufkommt, der bei uns 
schon Jahre anh\"alt.


\subsection{Was mu\ss{} ich zum \mm-Turnier mitbringen?}


\mm\ l\"auft auf {\em allen Atari STs\/} (nat\"urlich auch TTs) in niedriger und 
hoher Auf\/l\"osung. Jeder Spieler braucht nat\"urlich einen Computer --- das 
ist nicht so dramatisch, wie es zuerst klingt, jeder bringt einfach seinen 
mit. Pro Computer wird ein MIDI-Kabel\footnote{ein normales 5 poliges 
DIN-Kabel. {\em Achtung: kein Mono-Kabel!}} ben\"otigt. Dazu empfehlen wir 
einen Joystick, man kann aber auch mit Maus spielen. Au\ss{}erdem sollte jeder an 
gen\"ugend Steckdosen denken! An Kleidung sollte man selbstverst\"andlich das 
\mmI-T-Shirt anziehen (Im Winter dr\"ucken wir auch mal ein Auge zu\dots).


\subsection{Der Aufbau}


Erst einmal sollte man alle Rechner mit den MIDI-Kabeln verbinden. Dazu hat 
es sich als zweckm\"a\ss{}ig erwiesen, auf jeden Stecker einen Aufkleber "IN" 
bzw. "OUT" zu kleben --- dann kommt man nicht so leicht durcheinander. 
Dann steckt man einen Stecker in den MIDI-Out-Port des ersten Rechners und 
steckt das andere Ende in den MIDI-In-Port des zweiten Rechners. Von dem 
f\"uhrt nun wieder ein Kabel zum dritten Rechner usw. Den letzten Rechner 
verbindet man wieder mit dem ersten.


Alle Spieler laden nun \mm. Dann h\"ort man erst einmal den schicken 
Digi-Sound von J\"urgen Piscol. Dieser kann mit einer beliebigen Taste 
abgebrochen werden. Dr\"uckt man gleich nach dem Laden eine Taste, kann der 
Sound unterdr\"uckt werden. Die MIDI-\"ubertragung funktioniert nat\"urlich 
w\"ahrend der Musik noch {\em nicht\/}, da sie alle Rechenzeit verbraucht.


Nach dem Tastendruck wird ein Ringtest durchgef\"uhrt. Der letzte Spieler 
kommt automatisch in den Master-Modus --- wenn der Ring geschlossen ist, 
d.\,h., da\ss{} keine Kabel vertauscht oder vergessen wurden.


Falls kein Spieler in den Master-Modus kommt, ist der Ring fehlerhaft. Mit 
\verb|ALTERNATE-R| kann der Ringtest sonst noch einmal wiederholt werden: 
Wenn der Ring dann geschlossen ist, wird dieser Rechner zum Master, 
ansonsten bleibt er Slave. Wenn dies nicht hilft, dann sollte {\em einer\/} 
zwangsweise den Master-Modus aktivieren, indem er {\tt 
AL\-TER\-NATE}\verb|-~| dr\"uckt.


Der Master sollte dann eine Message durch den Ring schicken; dazu dr\"uckt 
er \verb|F1|. Wenn der Ring funktioniert, zeigt jeder Slave und der Master 
eine Copyright-Message an. Wenn irgendwo ein Kabel vergessen wurde oder 
falsch drinsteckt, wird {\em ab diesem Rechner\/} keine Message mehr 
angezeigt --- der Master erh\"alt einen "boo-boo". Dann sieht man sich 
einfach das Kabel von dem letzten Rechner mit Message und dem Folgerechner 
an. Wenn dort ordnungsgem\"a\ss{} ein Kabel steckt, sollte man es einfach 
austauschen.


\section{Die Einstellungen vor dem Spiel}


\subsection{Einstellungen f\"ur alle Rechner}


Solange man in dem Titelbild ist, kann jeder Spieler (Master oder Slave) 
folgende Tasten dr\"ucken:


Um mit Maus zu spielen, dr\"uckt man \verb|ALTERNATE-M|. Auf Joystick kann man 
mit \verb|ALTERNATE-J| zur\"uckschalten (Default ist Joystick).


Last not least kann das Spiel wieder verlassen werden. Wenn die anderen im 
Ring noch eine Partie daddeln wollen macht man sich damit nat\"urlich sehr 
unbeliebt, da die MIDI-Daten \"uber den eigenen Recher nicht weitergegeben 
werden.


\subsection{Der Bildschirmaufbau}


Nach dem Laden erscheint ein Bildschirm, der auf Farbe etwas anders aussieht 
als auf Monochrom; die einzelnen Anzeigen sind aber bei beiden an der 
gleichen Stelle. (Das Bild zeigt die hohe Auf\/l\"osung):

\bild{midipic}{Das Einschaltbild}


In der Mitte sieht man das Fenster zum Maze. W\"ahrend des Spiels hat man hier 
einen "3D-Blick". Oben links ist die Status-Anzeige (HOW AM I). Hier 
erscheint zu Spielbeginn ein strahlender Smiley, der immer ungl\"ucklicher 
schaut, je h\"aufiger man getroffen wurde. Rechts davon ist die Score-Anzeige. 
Jeder Spieler hat hier einen 3-stelligen Z\"ahler, der den aktuellen 
Punktestand anzeigt. Hier ist der entscheidenste Unterschied zwischen Farbe 
und Monochrom:


\begin{itemize}

\item Auf {\em Farbe\/} werden die ersten acht Spieler in acht verschiedenen 
Farben angezeigt. Augen und Mund sind schwarz, au\ss{}erdem sind sie schwarz 
umrandet. Wenn mehr als acht Spieler durchs Maze irren, wiederholen sich die 
Farben. Die zweiten acht haben dunkelrote Augen, Mund und Rand. Welche Farbe 
man selbst hat, sieht man an der Status-Anzeige links. Unter der 
Score-Anzeige sucht man dann die Farbleiste, die dieser Farbe entspricht. 
Wenn man einen schwarzen Rand hat, gilt der obere Z\"ahler, sonst der untere.


\item Auf {\em Monochrom\/} gab es mit der Farbgebung leider technische 
Probleme. Deshalb hat jeder Spieler hier ein anderes Muster; neben jedem 
Z\"ahler ist das Muster des Spielers dargestellt, zu dem er geh\"ort. Das eigene 
Muster sieht man wieder in der Status-Anzeige.

\end{itemize}


Links vom Fenster ist ein Kompa\ss{} und eine Schu\ss{}warnung (werden sp\"ater 
erkl\"art). Rechts neben dem Fenster sieht man vier Z\"ahler:


\begin{enumerate}

\item Die ersten drei werden nur zu statistischen Zwecken mitgef\"uhrt; sie 
sind f\"ur das Spiel unwichtig. Der {\em Hits}-Z\"ahler z\"ahlt alle deine 
Sch\"usse, die getroffen haben,


\item der {\em Kills}-Z\"ahler die Smileys, die du aus der Welt geschafft 
hast,


\item {\em Score\/} gibt alle Punkte an, die du in dieser Partie schon 
verbuchen konntest. Er sollte eigentlich $= \hbox{Hits} + 2 \cdot 
\hbox{Kills}$ sein, da ein Kill 3 Punkte und ein Hit 1 Punkt z\"ahlt.


\item {\em Money\/} ist da schon wichtiger: Er entspricht dem Score-Z\"ahler 
minus den schon ausgegebenen Punkten. Er zeigt also an, wieviele Punkte man 
noch in Extras investieren kann bzw. wieviele Punkte man schon f\"ur den Sieg 
verbuchen kann. Man gewinnt, wenn der Money-Z\"ahler den Win-Score erreicht 
hat. Dieser Z\"ahler ist derjenige, den man in der gro\ss{}en Scoreanzeige oben 
wiederfinden kann, wenn man Singles spielt.


Beim Teamplay sieht man in der gro\ss{}en Scoreanzeige die Summe aller 
Money-Z\"ahler aller Spieler des betreffenden Teams. Das Team, dessen 
Gesamtz\"ahler gleich dem Win-Score ist, hat gewonnen. Beim Kaufen ist nur der 
eigene Money-Z\"ahler von Bedeutung; man kann nicht mit fremder Leute Geld 
einkaufen.

\end{enumerate}


Unter dem Fenster ist die Pop-Chart. Hier werden die Gesichter der Spieler 
angezeigt, die du abgeschossen hast.


Unten ist die Kommentarzeile. Hier werden alle Textein- und -ausgaben 
get\"atigt. W\"ahrend des Spiels erscheint dort entweder der Name dessen, den 
man angeschossen hat oder der des unfreundlichen Zeitgenossen, der durch 
einen selbst seinen Hits-Z\"ahler erh\"ohen konnte. Ist gerade nix los, werden in
dieser Zeile die vorhandenen Extras ausgeben (durch ihre Tasten-Entsprechung).


\subsection{Der Spielstart, die Tastaturbelegung}


Mit \verb|F1| kann --- wie schon erw\"ahnt --- eine Copyright-Meldung \"uber den 
Ring schicken.


Der Master mu\ss{} ein Maze laden, in dem das Spiel stattfinden soll. \mm\ l\"adt 
automatisch "MIDIMAZE.MZE"; wenn man ein anderes haben will, dr\"uckt man 
\verb|F2|. Darauf erscheint ein File-Selector\footnote{was auch der Grund 
daf\"ur ist, da\ss{} der Master, im Gegensatz zu den Slaves, \mm\ nicht aus dem 
\texttt{AUTO}-Ordner laden kann}; nach dem Laden sieht der Master das Maze in 
einer \"ubersichtskarte.


Die Score-Anzeige aller Spieler kann mit \verb|F3| gel\"oscht werden (alle 
Tastenkommandos funktionieren nur beim Master, Slaves k\"onnen nat\"urlich 
nichts eingeben).


Mit \verb|F4| wird die Namenseingabe gestartet. Jeder Slave --- und der 
Master --- k\"onnen ihren Namen eingeben. Alle Zeichen sind zugelassen, 
\verb|Return| beendet die Eingabe, \verb|Esc| l\"oscht das Eingabefeld, die 
Cursortasten, \verb|Backspace| und \verb|Delete| haben ihre normale 
Bedeutung. Der Master kann erst dann weitermachen, wenn alle Spieler fertig 
sind. Dann erscheint bei jedem Spieler die Message "Names completed".


\verb|Return| startet der Master das eigentliche Spiel. Dann erh\"alt er 
erstmal die Masterpage (wenn der Ring in Ordnung ist):


\subsection{Die Masterpage}

\bild{masterp}{Voreinstellungen auf der Masterpage}


Diese Seite sieht --- zugegeben --- auf den ersten Blick reichlich chaotisch 
aus; das l\"a\ss{}t sich aber bei so vielen Einstellungsm\"oglichkeiten nicht 
vermeiden. Oben steht erstmal die Zahl der Computer im Ring.


Darunter steht f\"ur jeden Spieler eine Zeile. Links ist jedem ein 
Kennbuchstabe zugeordnet, dahinter steht der Name. Danach sehen Sie eine 
Zeile mit den Einstellungen f\"ur den jeweiligen Spieler. Oben ist ein 
Kennzeichen f\"ur die jeweilige Bedeutung ({\em Achtung:\/} 
Gro\ss{}/Kleinschreibung beachten!).


\begin{deflist}{Win}

\item[Win] ist der Score, der zum Gewinnen n\"otig ist. Ein Anschu\ss{} (Hit) 
bringt einen Punkt, ein Abschu\ss{} (Kill) drei. Wenn man einen Gegner zweimal 

an- und einmal abschie\ss{}t (normal, wenn keiner dazwischenfunkt) bringt einem 
das also f\"unf Punkte ein --- mit zehn Komplettabsch\"ussen gewinnt man, wenn 
man diesen Score nicht \"andert (Der default-Win-Score bei kleinem Extra-Satz 
ist 50).


\item[T] bezeichnet das Team, zu dem der betreffende Spieler geh\"ort. M\"oglich 
sind die Buchstaben A--H. Wenn alle Spieler im selben Team sind, wird 
"Singles" gespielt, d.\,h. jeder k\"ampft f\"ur sich allein.


\item[L] Reload Time --- die Zeit, die der Spieler nach einem Schu\ss{} braucht, 
bis wieder ein Schu\ss{} abgefeuert werden kann. Da\ss{} man schie\ss{}en kann, sieht 
man an einem sehr kleinen Fadenkreu\ss{} in der Mitte des Fensters, das nach 
einem Schu\ss{} verschwindet, bis nachgeladen worden ist. "A" steht f\"ur 0, je 
gr\"o\ss{}er der Buchstabe, desto l\"anger dauert es, bis man wieder schie\ss{}en kann. 
"A" ist also eine Art Maschinengewehr, man schiebt nur einen Schu\ss{} vor 
sich her, der absolut t\"odlich ist. Trotzdem ist, wie im ganzen \mm, nur ein 
Schu\ss{} gleichzeitig m\"oglich. Feuert man einen zweiten, so wird der erste 
gel\"oscht.


\item[F] Refresh Time --- die Zeit, nach der man ein Leben zur\"uckbekommt, 
wenn man angeschossen worden ist. Der Buchstabe ist \"ahnlich dem der Reload 
Time aufgebaut: je gr\"o\ss{}er der Buchstabe, desto l\"anger die Zeit.


\item[G] Regenerate Time --- die Zeit, nach der man wieder ins Leben 
zur\"uckgeholt wird, wenn man g\"anzlich abgeschossen worden ist. Diese Zeit 
sollte nicht zu gro\ss{} gew\"ahlt werden, schlie\ss{}lich langweilt sich ein 
nicht-aktiver Spieler nur, bis er wieder mitmachen darf.


\item[V] Revive Lives --- Die Zahl der Leben, die man nach einem Abschu\ss{} 
oder am Spielstart bekommt. Normalerweise bekommt man drei Leben; nach einem 
Anschu\ss{} wird auch nur auf drei "refreshed", man kann aber auch mehr Leben 
vergeben (0--9). Die Status-Anzeige zeigt trotzdem maximal drei Leben, 
sprich: den strahlenden Smiley. Hier kann man einen Spieler eine Pause 
machen lassen: man setzt diesen Wert auf 0; der Spieler ist immer tot und 
st\"ort das Spiel nicht.


\item[Z] Zig Zag --- tr\"agt man hier einen gr\"o\ss{}eren Wert als "A" ein (nicht 
\"uber\-trei\-ben!) f\"angt der Schu\ss{} an, zu eiern. Am besten ausprobieren!


\item[i] Invisibility (250~Punkte) --- wohl das h\"arteste Feature, da\ss{} \mm\ 
zu bieten hat. Der Spieler ist einfach nicht zu sehen und damit automatisch 
unverwundbar. Andere Smileys k\"onnen einfach durch ihn durchlaufen, ohne ihn 
zu bemerken (nicht aber umgekehrt).


\item[w] Walls (40~Punkte) --- der Spieler kann durch W\"ande schie\ss{}en. Der 
Schu\ss{} wird nur dann gestoppt, wenn er an die Au\ss{}enmauern des Maze st\"o\ss{}t. 
Walls vertr\"agt sich selbstverst\"andlich nicht mit dem Reflective Shot 
(letzteres wird ignoriert).


\item[h] Hide on Map (15~Punkte) --- der Spieler kann von anderen Spielern 
nicht auf der \"ubersichtskarte gesehen werden --- auch nicht, wenn diese 
\verb|F7| aktiviert haben.


\item[f] Fast Shot (20~Punkte) --- der Schu\ss{} des betreffenden Spielers ist 
doppelt so schnell wie der normale. Das sollte man nicht untersch\"atzen, ein 
Duell ist fast immer gewonnen, da die eigenen drei Sch\"usse den Gegner 
schneller erreichen als umgekehrt.


\item[a] Auto Answer (25~Punkte) --- wenn der Spieler, der im Genu\ss{} dieses 
Extras steht, von jemandem hinterr\"ucks angeschossen wird, r\"acht sich der 
erstere mit einem Schu\ss{}, der genau entgegengesetzt fliegt und damit den 
B\"osewicht mit gro\ss{}er Wahrscheinlichkeit trifft. Es ist nur etwas 
gew\"ohnungsbed\"urftig, da\ss{} der eigene Schu\ss{} pl\"otzlich verschwindet, wenn man 
angeschossen wurde. Denn auch hier gilt, wie im ganzen \mm: {\em Jeder 
Spieler kann immer nur einen Schu\ss{} zur Zeit abfeuern!\/} Startet man einen 
zweiten, verschwindet der erste automatisch.


\item[s] Schu\ss{}radius (25~Punkte) --- wenn diese Einstellung aktiv ist, ist 
der Schu\ss{} doppelt so gro\ss{}. Er sieht zwar normal aus, trifft aber eben noch, 
wenn er normalerweise schon vorbeigegangen w\"are.


\item[q] Quick (20~Punkte) --- Der Spieler wird doppelt so schnell. Auch das 
will ge\"ubt sein: Man hat anfangs etwas Schwierigkeiten, vern\"unftig um Eck 
en zu fahren.


\item[r] Reflective Shot (25~Punkte) --- der Schu\ss{} wird an der Wand nicht 
absorbiert, sondern reflektiert! Das l\"auft zwar nicht immer mathematisch 
exakt (der Algorithmus w\"are zu langsam), f\"urs Spiel reicht es aber aus. Auch 
hier ist zu beachten, da\ss{} nur ein Schu\ss{} zur Zeit m\"oglich ist. Witzig ist, 
da\ss{} man auch noch jemanden treffen kann, wenn man selbst im Jenseits weilt, 
der Schu\ss{} aber noch unterwegs ist.


\item[d] Deadly Fire (35~Punkte) --- dieser Schu\ss{} ist absolut t\"odlich; ein 
Treffer t\"otet jeden Spieler, egal wieviele Leben er noch hat. Allerdings 
bekommt man ein paar Punkte weniger, als bei normalem Spiel, weil man ja nur 
drei Punkte f\"ur einen Gegner erh\"alt (der Anschu\ss{} entf\"allt).


\item[n] No Deadly Fire (10~Punkte) --- kompensiert Deadly Fire. Es gilt nur 
f\"ur denjenigen, der dieses Extra besitzt; wenn dieser also von jemandem 
getroffen wird, der Deadly Fire besitzt, wird das als normaler Treffer 
gewertet.


\item[g] Got You (25~Punkte) --- jeder, der den Besitzer dieses Features 
anschie\ss{}t, ist ein wahrer Selbstm\"order: Er verliert sofort ein Leben. Wer 
ihn dreimal trifft, ist tot. Der Angeschossene erh\"alt die Punkte daf\"ur.


\item[b] No Got You (10~Punkte) --- das Anti-Feature dazu. Nur Besitzer von 
No Got You k\"onnen Leute mit Got You erledigen. Da die Anti-Features relativ 
billig sind, sollte man sich also vielleicht erstmal diese zulegen. Wenn 
jeder Got You hat und man selbst kein No Got You, dann wird das Spiel echt 
frustig.


\item[k] Key (1~Punkt) --- der Schl\"ussel erlaubt es einem, durch T\"uren zu 
laufen. Tip: Durch eine T\"ur fahren und \verb|F10| dr\"ucken --- jeder 
Verfolger wird total verwirrt.


\item[$<$] (Cursor links) (0~Punkte) --- erlaubt es dem Spieler, sich 
langsam zu drehen. Die zugeh\"orige Taste ist Cursor links. Damit kann man 
sehr sch\"on einen Gegner zwischen zwei Ecken anpeilen.


\item[$>$] (Cursor rechts) (0~Punkte) --- erlaubt schnellere Drehungen --- 
f\"ur den Fall, da\ss{} man schnell um Ecken mu\ss{}.


\item[Space] Karte (0~Punkte) --- schaltet die Karte ein bzw. aus. Sollte 
normalerweise jedem Spieler erlaubt sein. Damit \verb|Space| funktioniert, 
mu\ss{} zu\-s\"atz\-lich \verb|F2| angeschaltet sein (siehe dort)


\item[$\sim$] Durch W\"ande gehen (250~Punkte) --- der Spieler kann 
seelenruhig durch jede Wand rennen. Nur das Maze kann er nicht verlassen. 
Wenn er an den Rand kommt, kann er sich {\em in\/} die Au\ss{}enmauer stellen 
und halb heraussehen (sieht dort aber nur W\"ande).


\item[\#] No Shot --- verbietet dem Spieler das Schie\ss{}en. Das ist nat\"urlich 
nicht sehr fair, es ist haupts\"achlich f\"ur einen Midicam gedacht. Dieses 
Extra kann nicht gekauft werden, selbst wenn man es auf "k" oder "m" 
steht (siehe unten).


\item[c] Colorlessness (75~Punkte) --- Wer dieses Feature kauft, wird ein 
bi\ss{}chen bla\ss{} im Gesicht: Er wird durchsichtig, nur sein Rahmen und die Augen 
bleiben noch. Wenn ein Spieler etwas weiter entfernt steht, sieht man ihn 
fast nicht mehr. Im Gegensatz zu Invisibility bleibt man allerdings weiter 
verwundbar. Besonders wirksam ist dieses Extra zusammen mit \verb|F8|. Auch 
mit einem Reflective Shot macht es sich gut, da auch der Schu\ss{} nur als 
Umrandung zu erkennen ist. Nicht mehr so einfach ist jedoch das Erkennen der 
Teamzugeh\"origkeit (man neigt dann dazu, erstmal pr\"aventiv abzuschie\ss{}en).


\item[Space] ganz rechts ist noch ein Space, das nicht normal angesprochen 
werden kann. Es kennzeichnet einen Midicam (siehe unten).


\item[1] (0~Punkte) Mit \verb|F1| kann man sich umdrehen (180\grad). Das ist 
besonders sinnvoll, wenn man pl\"otzlich von hinten angeschossen wird. Damit 
man diese Funktion schneller erreichen kann, ist sie w\"ahrend des Spiels \"uber 
die 0 am Zehnerblock aktivierbar.


\item[2] (2~Punkte) Mit \verb|F2| erm\"oglicht man sich das Anzeigen der 
Karte. Das ist nicht mit \verb|Space| zu verwechseln:

\begin{itemize}

\item Mit \verb|F2| erlaubt man sich das Anzeigen (einmaliges Kaufen der 
Karte). Ohne aktiviertes \verb|F2| passiert bei \verb|Space| gar nichts.


\item Mit \verb|Space| kann man die Karte ein- und ausschalten, wenn man im 
Genu\ss{} von \verb|F2| steht.

\end{itemize}


\item[3] (2~Punkte) Mit \verb|F3| verhindert man, da\ss{} man von Teampartnern 
{\em ab\/}ge\-schos\-sen werden kann. Du wirst also von einem Partner 
angeschossen --- ein Leben weniger --- beim zweiten Schu\ss{} ebenso. Alle 
weiteren Sch\"usse (die t\"odlich w\"aren) sind jetzt aber wirkungslos. Immun ist 
aber nur der Spieler, der dieses Feature hat. Er kann nun weiterhin andere 
aus seinem Team, die es nicht besitzen, abschie\ss{}en. Wenn "Singles" 
gespielt wird, hat \verb|F3| keine Bedeutung.


\item[4] (5~Punkte) Mit \verb|F4| verhindert man zus\"atzlich, da\ss{} man von 
Teampartnern {\em an\/}ge\-schos\-sen werden kann. \verb|F3| ist dann 
unn\"otig.


\item[5] (6~Punkte) Mit \verb|F5| kommt man in den Vorzug eines Kompasses. 
Er wird links vom Fenster angezeigt. Das ist z.\,B. n\"utzlich, wenn man in 
einer Eck e von drei W\"anden umgeben ist, um sich gezielt in die Richtung 
des Ausgangs zu drehen (90\grad) und nicht andersherum (270\grad).


\item[7] (10~Punkte) Mit \verb|F7| erscheinen alle Gegner mit auf der Karte. 
Genau den gleichen Effekt hat man, wenn man sich die Karte ansieht, w\"ahrend 
man tot ist. Leute, die "Hide On Map" angeschaltet haben, sind aber auf 
jeden Fall auf der Karte unsichtbar.


\item[8] (15~Punkte) Mit \verb|F8| schaltet man sich das Gesicht ab. Die 
Gegner sind nat\"urlich im Nachteil, wenn sie nicht wissen, ob man sie 
ansieht.


\item[9] (5~Punkte) Mit \verb|F9| erlaubt man sich eine Schu\ss{}warnung. Wenn 
ein gegnerischer Schu\ss{} nahe an einem vorbeifliegt, leuchtet eine Zeit lang 
ein "S" im "Alert"-Feld auf. Das ist n\"utzlich, wenn man verfolgt wird, 
ohne es zu merken. Wenn der Gegner anf\"angt zu schie\ss{}en und zuerst vielleicht 
nur die Mauer trifft, ist man schon gewarnt.


\item[0] (5~Punkte) Mit \verb|F10| wird man sofort an eine andere Stelle im 
Maze "gebeamt". Als letzter Ausweg, wenn man schon zweimal getroffen 
worden ist, ist das sehr praktisch. W\"ahrend des Spiels auch \"uber 
\verb|Enter| zu erreichen.

\end{deflist}




\subsubsection{Wie stellt man nun was ein?}


Am linken Rand steht f\"ur jeden Spieler ein Kennbuchstabe. Man gibt nun 
diesen Buchstaben des Spielers ein, den man editieren m\"ochte. Seine Zeile 
wird darauf invertiert dargestellt. Die g\"angigsten Extra-Kombinationen sind 
\"uber Funktionstasten zu erreichen:


\begin{deflist}{F10}

\item[F1] f\"ur die \mmI~I-Einstellung. Das hei\ss{}t, man hat keine Extras und 
kann nur die Karte ansehen.


\item[F2] f\"ur die Normaleinstellung. Das hei\ss{}t, die Funktions- und 
Cursortasten sind belegt; mit \verb|K| kann man einen Schl\"ussel kaufen und 
mit \verb|R| einen Reflective Shot. Deadly und Gotcha sind mit Antifeatures 
erh\"altlich.


\item[F3] f\"ur die "Superman"-Einstellung. Man kann sich einfach alles 
aktivieren, jedes Extra ist kostenlos.


\item[F4] f\"ur den gro\ss{}en Extra-Satz. Alle Features sind kaufbar, mit 
Ausnahme von Invisibility, Walls, \verb|~| und \verb|#|.


\item[F5] f\"ur einen Midicam. Ein solcher Spieler ist unsichtbar, kann nicht 
schie\-\ss{}en und nicht getroffen werden. Er nimmt also praktisch nicht am 
Spiel teil, kann aber alle beobachten.

\end{deflist}


Diese Tasten dienen der schnellen Grobeinstellung. Man kann nun jedes 
einzelne Extra\footnote{einzige Ausnahme ist das Space ganz rechts; Midicams 
k\"onnen nur \"uber \texttt{F5} aktiviert werden} an- und abschalten. Dazu dr\"uckt 
man den Buchstaben des gew\"unschten Extras, so wie er oben am Bildschirm zu 
sehen ist --- {\em Achtung:\/} hier mu\ss{}t du also 1--9 (nicht die 
Funktionstasten), die $<$- und $>$-Tasten (nicht die Cursortasten) usw.\ 
dr\"ucken.


F\"ur jedes Feature gibt es verschiedene Status, die man durch dr\"ucken der 
entsprechenden Taste der Reihe nach durchschalten kann. Dabei bedeutet


\begin{deflist}{M}

\item[\_] Dieses Feature ist abgeschaltet. W\"ahrend des Spiels kann es nicht 
aktiviert werden.


\item[m] Dieses Feature ist m\"oglich. W\"ahrend des Spiels kann man es 
bekommen, indem man die entsprechende Taste dr\"uckt (dann wieder die 
\mbox{Funktions-,} Cursor- o.\,\"a. Taste). Durch einen erneuten Druck 
schaltet man es dann wieder ab usw. Ein- und Ausschalten ist kostenlos


\item[M] Dieses Feature ist m\"oglich und vorhanden. Mit einem Druck auf die 
entsprechende Taste kann man es ab-, dann wieder anschalten. Diese 
Einstellung entspricht also der obigen, nur ist das Extra hier am Spielstart 
schon aktiv.


\item[k] Dieses Extra ist kaufbar. Man kann es genauso ein- und ausschalten 
wie ein mit "m" gekennzeichnetes, nur kostet jedes {\em Ein\/}schalten 
hierbei etwas Score.


\item[K] Dieses Extra ist kaufbar, am Spielstart aber schon vorhanden. Wenn 
man es verkauft und erneut einkauft, kostet das Money.

\end{deflist}


Nachdem man die Einstellungen f\"ur den aktiven Spieler beendet hat, dr\"uckt 
man Cursor up oder down. Dann kann man den n\"achsten Spieler aktivieren. Mit 
\verb|SHIFT-A| kann man die Einstellungen des aktiven Spieler auf alle 
anderen Spieler kopieren. Mit \verb|SHIFT-C| und nachfolgender 
Spielerkennung kann man sie auf {\em einen\/} Spieler kopieren. Also z.\,B. 
stellt man f\"ur Spieler~A alles ein und dr\"uckt dann \verb|SHIFT-C B|. Darauf 
hat Spieler~B die gleichen Einstellungen. Wenn man noch auf einen anderen 
Spieler kopieren will, mu\ss{} wieder \verb|SHIFT-C| gedr\"uckt werden. Mit einer 
anderen Taste als einem Kennbuchstaben kann das Kopieren abgebrochen werden.


Mit einem erneuten Druck auf \verb|Return| startet man das Spiel. Mit 
\verb|Undo| restauriert man nach einem Spiel die Einstellungen, die man vor 
dem Spiel get\"atigt hatte. Ein Beispiel: Der Master hat einen Reflective Shot 
erlaubt (mit "k"); ein Spieler hat sich gierig einen gekauft. Nach dem 
Spiel bleiben diese Einstellungen erhalten; bei diesem Spieler steht nun ein 
gro\ss{}es "K". Starte der Master das Spiel einfach neu, h\"atte dieser Spieler 
von Anfang an einen Reflective Shot. Deshalb kann er mit \verb|Undo| die 
Einstellungen zur\"ucksetzen.


Beim Spielstart wird nochmals der Ring auf Funktionst\"uchtigkeit gepr\"uft. Das 
Maze wird vom Master an die Slaves \"ubertragen (beim weiteren Spiel mit dem 
gleichen Maze wird es nicht neu \"ubertragen). Wenn das Maze zu klein ist, um 
alle Smileys darin unterzubringen, wird mit einer entsprechenden 
Fehlermeldung abgebrochen.


\vfill

\pagebreak


\section{Feature\"ubersicht}


Der \"ubersichtlichkeit halber hier noch einmal alle Features mit Taste und 
Kosten:


\begin{center}

\begin{tabular}{|r|c|l|}

\hline

Karte schalten & & 0\\

180\grad drehen & \verb|F1| & 0\\

Karte erm\"oglichen &\verb|F2| & 2\\

Deadly Fire & \verb|F3| & 2\\

Friendly Fire & \verb|F4| & 5\\

Kompa\ss{} & \verb|F5| & 6\\

Gegner auf Karte & \verb|F7| & 10\\

Kein Gesicht & \verb|F8| & 15\\

Schu\ss{}warnung & \verb|F9| & 5\\

Hyperjump & \verb|F10| & 5\\

Reflective Shot & \verb|r| & 25\\

Quick & \verb|q| & 20\\

Invisibility & \verb|i| & 250\\

Durch W\"ande schie\ss{}en & \verb|w| & 40\\

Hide Player & \verb|h| & 15\\

Fast Shot & \verb|f| & 20\\

Auto Answer & \verb|a| & 25\\

Big Shot & \verb|s| & 25\\

Deadly & \verb|d| & 35\\

No Deadly & \verb|n| & 10\\

Got You & \verb|g| & 25\\

No Got You & \verb|b| & 10\\

Durch W\"ande gehen & \verb|~| & 250\\

Langsames Drehen & \verb|<| & 0\\

Schnelles Drehen & \verb|>| & 0\\

Colorlessness & \verb|c| & 75\\

Schl\"ussel & \verb|k| & 1\\

\hline

\end{tabular}

\end{center}

\end{document}

